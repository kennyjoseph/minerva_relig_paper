\section{Related work}

\subsection{Overview of the Arab Spring}

A host of historical factors led to the conditions in 2011 that made many Arab countries ripe for protest \citep{gelvin_arab_2015}. One long-standing issue was the increasing extent of economic problems caused by ineffective, corrupt and state-run economies.  These issues led to elevated levels of unemployment and high inflation \citep{dewey_impact_2012} as well as to food shortages and massive hikes in food prices -- all of which contributed to high levels of civil unrest \citep{goldstone_cross-class_2011,comunello_will_2012}.  The effect of high unemployment rates was particularly a problem because it heavily affected young, well-educated populations, i.e. individuals who had often been promised jobs as a result of their education \citep{dewey_impact_2012,gelvin_arab_2015}. This, combined with a ``youth bulge'' in which a disproportionate percentage of the population was  between the ages of 15-29 in many of the MENA region countries, provided a fodder of civil unrest that required only a spark to ignite and a gust of wind to spread throughout the region. This spark came, as noted above, in the form of Mohamed Bouazizi.  While Bouazizi may have provided the spark, it is generally agreed upon that social media, or more aptly, the existence of communication infrastructures that supported all forms of new media \citep{wolfsfeld_social_2013,tufekci_social_2012}, served as the wind to fan the flames to Egypt, Libya and several other countries in the Arab world. In addition to work focusing on why the revolutions occurred, several scholars have developed rationals for why some revolutions succeeded (e.g., in Egypt, Libya and Tunisia) whereas others failed (e.g., in Bahrain and Saudi Arabia) and still others remain unresolved (e.g., in Syria). In interviews, Gerald Feierstein, Assistant Secretary for Near Eastern Affairs at the U.S. State Department, attributes the variation of success in the Arab Spring to the strength of each countries’ civil society (REMOVED FOR BLIND REVIEW, 2015).

% Even before the protests, however, the existence of internet connectivity allowed individuals within the Arab region to observe the democratic processes existent in other regions of the world, stoking their desire to live in that type of environment \cite{hussain_what_2013}. During the early moments of the revolutions, these tools allowed both deliberate diffusion processes, defined as those ``carried out via the conscious sharing of tactics and frames by activists who are linked by networks that may be transnational'' and demonstration diffusion effects,  `` ' the power of precedent' '' to occur, rapidly engulfing an increasing number of actors in an increasing number of nations in protest \citep[][pg. 140]{bellin_reconsidering_2012}.  Through both logics, a further cause of spread was the strengthening of an underlying Arabic identity, which united protesters in their unified goal of empowerment and the ending of corruption. 

Other scholars have attributed the structure of governance to be the deciding factor in success or failure of the Arab Spring revolutions.  \citeauthor{goldstone_bringing_2013} \citeyearpar{goldstone_bringing_2013} suggests that chief amongst the predictors of success is the structure of the ruling regime.  Goldstone defines a \emph{personalist} regime as one in which a single individual – who may have begun as an elected leader, or head of a military or even party regime – takes total or nearly total control of the national government. In his work he provides qualitative evidence that ``the single best key to where regimes in MENA have been overturned or faced massive rebellions is where personalist regimes have arisen'' (pg. 14). \citeapos{goldstone_bringing_2013} work argues that personalist regimes were the most susceptible because their power was tied to the leaders' ability to provide the necessary economic and political incentives to their constituency, particularly in rentier countries that depended on oil production. 

The status of economic conditions played a role not only in fomenting revolution but also in different levels of success, as economic challenges in countries like Tunisia prevented personalist regimes from being able to “buy their way out” of the protests. It is likely that personalist regimes needed to retain their power over the military more than any other segment of society\cite{comunello_will_2012,battera_perspectives_2014}.  \cite{bellin_reconsidering_2012} argues that the relationship between revolutionary success can be narrowed down to the relationship between the ruling regime and the military - whether or not the regime was able to convince the military to shoot at the protestors and whether the soldiers carried out these orders. In countries where the military made the decision not to use force, protesters were able to flood the streets without fear of the full wrath of the state. This lack of impunity emboldened tougher protests that eventually led to the downfall of the Tunisian and Egyptian regimes. In contrast, in Saudi Arabia, the military opted to side with the regime and use force to quell protests, regimes managed to maintain power\footnote{After the protests leading to the Fall of the Shah of Iran, inconsistent military response can also have deleterious effects. Thus the middle category of occasionally violent occasionally peaceful responses by the military also emboldens protesters and can destablize the regime \cite{sick_all_1985}}.
%\footnote{While regimes certainly had other forces willing to deal with the protestors, the extent and breadth of the protests eventually came to a point at which it ``is sufficient to look at the character of the military and its capacity and will to repress in order to reckon the immediate chances of regime survival.'' \cite{bellin_reconsidering_2012}(pg. xx).}

In making the decision of whether or not to fire on protestors, one important factor was  the extent to which previously disparate social groups formed a cross-class coalition in their protests and revolutionary efforts \cite{goldstone_cross-class_2011}.  A unified coalition of protestors made it more difficult for the military to justify the use of force in their response to the protest for two reasons. First, the combination of various social groups lessened the military's ability to claim that violent actions were a response to a particular out-group in the interest of protecting the ``nation''.  Second, the sheer size of such a coalition would relegate military action to being viewed as ``illegitimate slaughter'' \cite[][pg. 132]{bellin_reconsidering_2012}.  

The development of these unified coalitions across various social groups also provided an opportunity for the news media to characterize protestors under a national identity, rather than as protests via one specific subgroup.  As we discuss shortly, the portrayals of the revolutions by news media thus had an important impact on revolutionary outcomes in that their coverage ultimately influenced a final factor in revolutionary success, which was the extent and type of involvement in the revolutions in the various nations from international powers \cite{goldstone_bringing_2013,comunello_will_2012}.  This applies both to the actions of the West during the revolutions as well as the actions taken by Arab nations themselves.  

Social media had much the same effect, increasing both the level of information spread and the extent to which individuals felt compelled to participate in protests \citep{tufekci_social_2012,wolfsfeld_social_2013,bellin_reconsidering_2012}.   Similarly, new social media may have provided a forum for national identities to emerge \citep{cottle_media_2011}.  More generally, however, there were important similarities between news media and social media not only in that they served similar purposes, but that they existed within a symbiotic relationship throughout the Arab Spring \cite{cottle_media_2011}.  As we will see, the intertwinement of news media content and content from social media was significant, but existed more heavily within discussions related to particular themes and particular countries.
%  The authors consider how, across 8 different properties of social media, technological determinists (people who believed social media played \emph{the} causal factor) and the techno-realists (people who believed social media played \emph{no} causal role) differ on their opinions regarding the effect of social media.  %Additionally, \cite{hussain_what_2013} provide a unique analysis of the role that communication infrastructures played in coordination with other effects on the revolutions. 


\subsection{Using new media to study the Arab Spring}

Having intimated our position on whether or not social media played a role in the revolutions, we now turn to how data from social media has been used to better understand the processes inherent to the revolutions as a whole. In particular, we focus on Twitter data in the present study. The earliest work that examines the relationship between new media and the Arab Spring was that of  \cite{lotan_revolutions_2011}. The authors stressed the strong interplay between Twitter and news media, and showed that the relationships between mainstream media outlets, activists, journalists and bloggers differed in datasets collected from Tunisia and Egypt. \cite{lotan_revolutions_2011} considered data that used the hashtags \#egypt or \#libya, finding that Egypt and Libya displayed differing amounts of tweets in the two major languages of interest, Arabic and English.  Egypt had significantly more tweets in Arabic than in English, and in Libya the reverse was true. Subsequent work by \cite{bruns_arab_2013} observed that the English-speaking world lost interest in the events of the Arab Spring much earlier than those tweeting in Arabic.  (AUTHOR, REMOVED FOR BLIND REVIEW) in their examination of the Benghazi consulate and Cairo embassy attacks set against the context of the Arab Spring, observed that news attended less to Libya than to Egypt, that the volume of tweets was lowest in Libya than in Egypt, and that there was no connection between Arabic and the non Arabic tweets.  As we focus largely on English-language keywords, such language effects must be kept in mind.

Perhaps most relevant to the present work, however, are the efforts of \cite{borge-holthoefer_content_2014}. These authors used Twitter data to test the extent to which individuals switched between Secularist and Islamist and pro and anti-military ``camps'' in Egypt during the Arab Spring.  While the authors focused on Arabic tweets, their efforts show that Twitter data provides a unique lens through which important and interesting social processes relevant to the revolutions can be studied. We extend their efforts to new questions, new data and new methods in the present work.
%\cite{bruns_arab_2013} work considers usage patterns in tweets that had the hashtags \#egypt or \#libya over the course of ten months in 2011.  

%One of the earliest works to focus on the use of Twitter during the Arab Spring was that of \cite{lotan_revolutions_2011}. This observation 

%Information structures differed across countries 


%\cite{lotan_revolutions_2011}
%For example, journalists and activists in both cases appeared to serve as primary information sources, with their efforts being retweeted by a variety of other types of actors.    
%These interested parties fall into roughly three categories:
%1. People directly connected to an incident, either as residents or expatriates that want to
%know about dangerous conditions and the state of their homes and families, or who are
%experiencing a crisis event firsthand;
%2. MSM who want to learn about developments on the ground so that they can provide upto-
%date coverage across media channels and hold audience attention; and
%3. General interest readers who want to know about events as they happen.

%\cite{bruns_arab_2013}   They found, consistent with a variety of previous work, that Arabic and English speaking users mostly interacted amongst themselves, and thus that there existed only a few actors who bridged the two communities.    This suggests that international interests varied and that within these nations there was varying levels of use of Twitter.  Finally, 


	%(hatem) 
%		Things that twitter was used for:
%		    Symbolic language (majority)
%			meet here ... (minority)
%		    documenting repression... (minority)
%		Important to note that people in Egypt knew something was going to happen weeks before the 25th ... something in Cairo is gonna happen 
	
\subsection{The role of the news media}

As argued by \cite[][pg. 60]{hussain_what_2013}, international news media organizations were vital during Arab Spring as they focused world-wide attention to the events in the region, helping to ``stave off overtly violent reactions from security forces.''  \cite[][pg. 656]{cottle_media_2011} argues that this was particularly important in ``alerting world opinion to repressive and potentially prosecutable acts of inhumanity''.  We would expect that the degree of coverage for particular topics, especially relating to violence and revolution, should be causal or correlated to the success and/or failure of uprisings across the Arab world. Other work by (REMOVED FOR BLIND REVIEW) has considered quantitative approaches to using news media data to better understand the Arab Spring. Specifically, the authors use news media to instantiate a dynamic network agent-based model and use model output to predict successful revolutions. While their use of newspaper data chiefly incorporates off-the-shelf statistical tools, it is an important indicator that news media data can be used in large quantities to provide accurate models of underlying social processes.

%\cite[][pg. 457]{goldstone_cross-class_2011} argues in a similar fashion that the extent to which the news media portrayed ``protestors as representative of the whole society, rather than as one particular group seeking partisan advantages for itself'' was likely to impact the success of the revolution.  %However, \cite{goldstone_cross-class_2011} notes that beyond the initial successes of the revolution, the extent to which national identities were retained in representations of the country could have important but unclear effects on the future of countries within the region. 


%
%Difference between really biased newspapers ... who said nothing was really happening ... and the international news who could cover it reasonably
%
%Egyptians relied heavily on rumors ... when you have a situation where everyone knows the news is propaganda .. you rely on rumors more  
%
%    Also when the media is obviously lying ... I'm looking at this streety as its being recorded ... the typical egyptian deligimizes the news
%
%	they also get a lot of international news from satellite TV ... 


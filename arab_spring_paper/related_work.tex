\section{Related work}

The immense number of complexities and historical artifacts that played a role in the beginnings, successes and failures of the Arab Spring prohibit a full exploration here.  Instead, we provide only a summary of some of the more widely accepted factors, and ones that readers should be aware of and keep in mind throughout our analysis. For a more detailed overview, we direct the reader to \citeapos{gelvin_arab_2015} recent book. In this section, we also discuss recent work utilizing social and/or news media during the Arab Spring, detailing how the methods and data used here are both similar and different.  

\subsection{Causal factors of the Arab Spring}

Two broad classes of causal factors can be considered - those that prompted revolution, and those which affected the success or failure of revolutions.  We briefly review each set of factors here.

\subsubsection{Causes of protests and their spread}

A host of historical factors led to the conditions in 2011 that made many nations in the Arab world ripe for protest \cite{gelvin_arab_2015}. One long-standing issue was the increasing extent of economic problems caused by ineffective, corrupt and state-run economies.  These issues led to high levels of unemployment and inflation \cite{dewey_impact_2012} as well as to both food shortages and huge hikes in food prices, all of which contributed to high levels of civil unrest \cite{goldstone_cross-class_2011,comunello_will_2012}.  The effect of high unemployment rates was particularly a problem because its effect was particularly strong on well-educated youth populations, individuals who had often been promised that their education efforts would be rewarded with jobs \cite{dewey_impact_2012,gelvin_arab_2015}. This, combined with a ``youth buldge'' in which a disproportionate percentage of the population was  between the ages of 15-29 in many of the MENA region countries, provided a fodder of civil unrest that required only a spark to ignite and a gust of wind to spread throughout the region.

This spark came, as noted above, in the form of Mohamed Bouazizi.  While Bouazizi may have provided the spark, it is generally aggreed upon that social media, or more aptly, the existence of communication infrastructures that supported all forms of new media\citep{wolfsfeld_social_2013,tufekci_social_2012}, served as the wind to spread the flame to Egypt, to Libya and on to several other nations in the Arab world.  Even before the protests, however, the existence of internet connectivity allowed individuals within the Arab region to observe the democratic processes existent in other regions of the world, stoking their desire to live in that type of environment \cite{hussain_what_2013}. During the early moments of the revolutions, these tools allowed both deliberate diffusion processes, defined as those ``carried out via the conscious sharing of tactics and frames by activists who are linked by networks that may be transnational'' and demonstration diffusion effects,  `` ' the power of precedent' '' to occur, rapidly engulfing an increasing number of actors in an increasing number of nations in protest \citep{}.  Through both logics, a further cause of spread was the strengthening of an underlying Arabic identity, which united protesters in their unified goal of empowerment and the ending of corruption. 
	
\subsubsection{Causes of Varying Outcomes}

\citeauthor{goldstone_bringing_2013} \citeyearpar{goldstone_bringing_2013} suggests that several additional predictors were important in determining whether or not a particular government was ultimately overthrown.  Chief amongst these predictors, he argued, was the structure of the overarching regime had a strong impact on whether or not the government was ultimately overthrown.  This view is a piece of a consistent take on  the properties of the ruling regime as the primary factor in the success of a particular revolution\cite{bellin_reconsidering_2012,comunello_will_2012,goldstone_bringing_2013}.  Goldstone defines a \emph{personalist} regime as one in which a single individual – who may have begun as an elected leader, or head of a military or even party regime – takes total or nearly total control of the national government. He then provides qualitative evidence for his believe that ``the single best key to where regimes in MENA have been overturned or faced massive rebellions is where personalist regimes have arisen''.

\citeapos{goldstone_bringing_2013} work argues that personalist regimes were the most susceptible because their power was tied to their ability to provide the necessary economic and political incentives to their constituency, particularly in nations that depended on oil production.  While \cite{comunello_will_2012} notes that personalist, or as he refers to them, neopatrimonial states, had controlling arms that made it difficult to organize any sort of formal protest, three other factors led to conditions in which such formal protests could arise.

One such factor that has already been discussed was the economic conditions under which the revolution occurred. While these economics played a role in bringing about revolution, they also prevented personalist regimes from being about to ``buy their way out'' of the protests, and thus also had a role in the revolutions' successes. The second factor that played into the success of personalist regimes in the face of revolution was the relationship of the regime to the military \cite{comunello_will_2012,battera_perspectives_2014}.  This relationship can be boiled down to one decision the military made - whether or not to shoot at the protestors \cite{bellin_reconsidering_2012}\footnote{While regimes certainly had other forces willing to deal with the protestors, the extent and breadth of the protests eventually came to a point at which it ``is sufficient to look at the character of the military and its capacity and will to repress in order to reckon the immediate chances of regime survival.'' \cite{bellin_reconsidering_2012}(pg. xx).}.  In countries where the military made the decision to quell protests with violent force, protesters were able to flood the streets without fear of the full wrath of the state. This lack of impunity led to stronger protests that eventually led to the downfall of the Tunisian and Egyptian regimes.  In contrast, in Saudi Arabia, for example, where the military opted to side with the regime and use force to quell protests, regimes managed to maintain power.

In making the decision of whether or not to fire on protestors, one important factor was  the extent to which previously disparate social groups formed a cross-class coalition  in their protests and revolutionary efforts \cite{goldstone_cross-class_2011}.  A unified coalition of protestors made it more difficult for the military to justify the use of force in their response to the protest for two reasons. First, the combination of various social groups lessened the military's ability to claim that violent actions were a response to a particular out-group in the interest of protecting the ``nation''.  Second, the sheer size of such a coalition would relegate military action to being viewed as ``illigitimate slaughter'' \cite{bellin_reconsidering_2012}.  

The development of these unified coalitions across various social groups also provided an opportunity for the news media to characterize protestors under a national identity, rather than as protests via one specific subgroup.  As we discuss shortly, the portrayals of the revolutions by news media thus had an important impact on revolutionary outcomes in that their coverage ultimately influenced a final factor in revolutionary success, which was the extent and type of involvment in the revolutions in the various nations from international powers \cite{goldstone_bringing_2013,comunello_will_2012}.  This applies both to the actions of the West during the revolutions as well as the actions taken by Arab nations themselves.  

Social media, of course, had much the same effect, increasing both the level of information spread and the extent to which individuals felt compelled to participate in protests \cite{tufekci_social_2012,wolfsfeld_social_2013,bellin_reconsidering_2012}. As insinuated above, it is generally agreed upon that social media, and new media more generally, played at least some role in the spread and success of certain revolutions that occurred during the Arab Spring.  There do exist, however, many scholars who feel this effect has been overstated, or that no such effect exists at all. \cite{comunello_will_2012} give an overview of a significant amount of work focusing on the relationship between social media and the Arab Spring.  The authors consider how, across 8 different properties of socail media, technological determinists (people who believed social media played \emph{the} causal factor) and the techno-realists (people who believed social media played \emph{no} causal role) differ on their opinions regarding the effect of social media.  Additionally, \cite{hussain_what_2013} provide a unique analysis of the role that communication infrastructures played in coordination with other effects on the revolutions.

\subsection{\todoKenny{Using new media to study the Arab Spring}}

Having given our stance on the question of whether or not social media played a role in the revolution, we now turn to how data from social media has been used to better understand the processes inherent to the revolution as a whole.  In particular, we focus on Twitter, as this is the data available to us in the present study.  Of the many studies which have focused on Twitter in the context of the Arab Spring, we here highlight X facts

\cite{bruns_arab_2013} work considers usage patterns in tweets that had the hashtags \#egypt or \#libya over the course of ten months in 2011.  

One of the earliest works to focus on the use of Twitter during the Arab Spring was that of \cite{lotan_revolutions_2011}. This observation 

Information structures differed across countries 

	\cite{lotan_revolutions_2011}
		 The authors stress the strong interplay between Twitter and news media, and showed that the relationships between mainstream media outlets, activists, journalists and bloggers differed in datasets collected from Tunisia and Egypt. 

	\cite{bruns_arab_2013}
		Further, they observed that Egypt and Libya displayed differing amounts of tweets in the two languages, with Egypt having significantly more tweets in Arabic than English, and Libya the other way around.

\cite{lotan_revolutions_2011}
For example, journalists and activists in both cases appeared to serve as primary information sources, with their efforts being retweeted by a variety of other types of actors.    
These interested parties fall into roughly three categories:
1. People directly connected to an incident, either as residents or expatriates that want to
know about dangerous conditions and the state of their homes and families, or who are
experiencing a crisis event firsthand;
2. MSM who want to learn about developments on the ground so that they can provide upto-
date coverage across media channels and hold audience attention; and
3. General interest readers who want to know about events as they happen.

\cite{bruns_arab_2013}   They found, consistent with a variety of previous work, that Arabic and English speaking users mostly interacted amongst themselves, and thus that there existed only a few actors who bridged the two communities.    This suggests that international interests varied and that within these nations there was varying levels of use of Twitter.  Finally, \cite{bruns_arab_2013} observed that these patterns changed over time, with the English-speaking world gradually losing interest. 	

\cite{borge-holthoefer_content_2014}

	We apply our methodology on a set of nearly 6 million Arabic tweets crawled between June 21 and September 30, 2013. From these, we reconstructed the network of Twitter users who authored tweets in the collection, and we crawled all meta information for 120,000 users along with their latest 3,200 tweets prior to December 2013. All the tweets and Twitter user IDs in our collection are publicly available[41], so that researchers can replicate the exact dataset we used for possible future studies.
	
	We look at switching between Secularist and Islamist camps and between pro-military and anti-military camps. Our network and content analyses indicate that less than 5\% of users switched sides. Instead, the main narrative seems to be one of pro-military intervention and Secular users being dominant in terms of volume leading up to July 3, and anti-military intervention and Islamist users gaining in volume afterwards. Furthermore, in contradiction to the dominating narrative in mass media, the correlation between being a secular and a supporter of military intervention is far from perfect. However, some correlation was noticed between being an Islamist and against the military intervention.
	
	we focus on Arabic tweets
	
	
Many, many other studies on Twitter. Methods discussed in their context as appropriate.

	%(hatem) 
%		Things that twitter was used for:
%		    Symbolic language (majority)
%			meet here ... (minority)
%		    documenting repression... (minority)
%		Important to note that people in Egypt knew something was going to happen weeks before the 25th ... something in Cairo is gonna happen 
	
\subsection{\todoKenny{The role of the news media}}

(hatem)

Difference between really biased newspapers ... who said nothing was really happening ... and the international news who could cover it reasonably

Egyptians relied heavily on rumors ... when you have a situation where everyone knows the news is propaganda .. you rely on rumors more  

    Also when the media is obviously lying ... I'm looking at this streety as its being recorded ... the typical egyptian deligimizes the news

	they also get a lot of international news from satellite TV ... 

\cite{hussain_what_2013}
 international news organizations played in giving them the global exposure
to help stave off overtly violent reactions from security forces.

\citep{goldstone_cross-class_2011}

The impact
of public media in favor of the protestors is also greater if media representation shows
protestors as representative of the whole society, rather than as one particular group
seeking partisan advantages for itself.
For these reasons, virtually all successful revolutions



\section{Introduction}

Much has been made of the use of social media during the Arab Spring. At this point, it is almost common knowledge that the emphasis on social media as a \emph{cause} of the revolutions is overblown \citep{bruns_arab_2013,goldstone_bringing_2013} . However, many scholars have pointed to the fact that the use of social media may have aided certain aspects of the revolutions in important ways for different people  \citep{galle_who_2013,starbird_how_2012,papacharissi_affective_2012} and that data from social media may help to understand how the revolutions grew and spread \citep{lotan_revolutions_2011,bruns_arab_2013}. By the same token, recent work has suggested that data from news media during the time of the Arab Spring may have itself been indicative of important developments during that time period \cite{joseph_arab_2014,pfeffer_rapid_2012}.

The present work considers behavior on social media and news indicators as two predictors in a more general framework of instability. We consider 3 levels of predictors...

We use increasing nuance w/ Twitter/News predictors...

Our work provides a more nuanced view of the conclusions that can be reached via the analysis of both social media and newspaper data in the context of the underlying social and political climates of the region. 

... This finding suggests that high-level analyses of things like the number of tweets from a given country and tweets relevant to a particular word are only useful after careful consideration of the variety of biases inherent in this sort of data and of other important covariates.

\subsection{My list of covariates}

\begin{enumerate}
    \item {\bf Regime-based}
        \begin{enumerate}
            \item black-out of media vs not black out media 
            \item religious freedom indices
            \item personalist/not \cite{goldstone_bringing_2013}
            \item wealth (particularly oil) \cite{goldstone_bringing_2013}
            \item Twitter/News mentions of regime (of some kind) - sentiment?
            \item Twitter/News revolution/insurgent/violence or adaptation terms
        \end{enumerate}
    \item {\bf International relations} 
        \begin{enumerate}
            \item international network position
            \item international Twitter/News network position
        \end{enumerate}
    \item {\bf Population indicators } 
        \begin{enumerate}
            \item Twitter/news geospatial spread
            \item Religious indices
            \item Ethnic diversity / group indicators
            \item Twitter/news mentions of ethnic group
            \item Twitter/news attention / network position
            \item Twitter/news ``class'' cohesiveness \cite{goldstone_cross-class_2011}
            \item Twitter/news ``class'' spread \cite{goldstone_cross-class_2011}
        \end{enumerate}
\end{enumerate}

\subsection{Research Questions}

\begin{enumerate}
    \item {\bf Prediction task: Predict from combination of: }
        \begin{enumerate}
            \item Did revolution happen? Y/N
            \item Was revolution quelled? Y/N
            \item Is it still ongoing - did anything change? Y/N
        \end{enumerate}

    \item {\bf Prediction task: Predict level of instability}
\end{enumerate}

\subsection{Outstanding todos/questions}

\begin{enumerate}
    \item Most indicators come from news - how do we assess impact of news?
    \item What is a reasonable dependent variable for stability?
    \item Figure out what is up with that jump in the Twitter data
    \item Run Netmapper/get news networks
\end{enumerate}




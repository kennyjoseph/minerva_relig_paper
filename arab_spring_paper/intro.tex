\section{Introduction}
On December 17th, 2010, Mohamed Bouazizi immolated himself in Sidi Bouzid, Tunisia in response to harassment from both a local policewoman and local municipality officers.  Though Bouazizi was not the first to engage in this form of protest, for some reason his case resounded with others who themselves took to the streets in protest of constant harassment and victimization by a corrupt government.  Although early protests were relatively small and were met with violence from government forces, social media sites like Twitter, Facebook and YouTube were used to record these events and display them to the broader public.  These events are widely considered to be the beginning of what has come to be known, for better or worse \citep{gelvin_arab_2015}, as the Arab Spring.

There is little doubt that social media, and new media \citep{baym_personal_2010} more generally, played an important role in Arab Spring. However, it is almost common knowledge at this point that popular emphasis on social media as \emph{the} cause of the revolutions is overblown \citep{bruns_arab_2013,goldstone_bringing_2013,comunello_will_2012}. Recent research has thus instead focused on how social media may have aided certain aspects of the revolutions in important ways for different people  \citep{galle_who_2013,starbird_how_2012,lotan_revolutions_2011,tufekci_social_2012}, and in how social processes that were carried out via new media are reflective of those that occurred ``offline'' \citep{comunello_will_2012}. Similarly, recent work has suggested that data from newspaper articles written during the time of the Arab Spring also may be of use in better understanding these processes \citep{joseph_arab_2014,pfeffer_rapid_2012}. Thus, social media and the coverage of news media should be seen as both pieces and reflections of a complex system of causal structures that were at play.  

To date, however, prior work has largely considered how the news media or Twitter are useful in understanding the social processes at play during particular events across a small set of nations \citep{} or a series of events in a particular nation, most often Egypt \citep{}.  Further, few studies have considered news and Twitter data side-by-side, leaving questions as to the similar and different ways in which these media responded to different long and short term social processes during the Arab Spring. 

In the present work, we use a corpora of around 70 M tweets and around 700K newspaper articles to provide an overview of the change in topical focus over time in sixteen nations relevant to the Arab Spring.  We take a breadth-over-depth approach, attempting to reconcile patterns in Twitter usage and news media coverage over a wide range of countries and time periods. Additionally, we utilize the same methodology for both datasets, allowing us to compare results across media.  Specifically, we here focus on the following three research questions:
\begin{itemize}
	\item {\bf RQ1:} How did the topical focii of our news and Twitter data differ over time and across different nations?
	\item {\bf RQ2:} How did (un)successful government overthrows change the topical focii of news and Twitter data?
	\item {\bf RQ3:} Can we develop new hypotheses based on our data for the relationship between topical focii in news and Twitter data and (a lack of) social change? a holistic view of the topical focii of 
\end{itemize}

In order to study these three research questions, we begin by developing a set of N human-curated topical themes of interest based on prior literature and identifying terms that, when mentioned, are relevant to these themes.  We then search for these terms across all of our Twitter and newspaper data.  Where we find a term used in a particular tweet or news article, we determine the time at which the content was produced and the particular nation(s) in the Arab World that the content is relevant to.  We thus are left with a set of counts, over time, of the discussion of our different thematic content in different nations in the Arab world.

Na\"{i}vely, we could then use this count data, or rates directly calculated from this count data, to address our research questions.  We could compare normalized rates of usage across news and Twitter data to address RQ1, look at changes in these rates before and after government overthrows to address RQ2 and develop a case study to explore the changes in these rates in RQ3. As detailed in recent work by \cite{eisenstein_diffusion_2014}, and as we will show in this article, however, the direct utilization of count data, or rate data based off these counts, is a methodologically unsound decision. 

As \cite{eisenstein_diffusion_2014} discuss for Twitter, term count may be biased by unknown irregularities in the way Twitter provides tweets through its API \citep{morstatter_is_2013} or unique properties of the keyword or spatial queries researchers construct to obtain data from the API \citep{joseph_approach_2014}.  Similarly, superfluous coverage by news media on particular nations may lead to artificial increases in counts or, if focused on themes not of interest in the present work, superfluous decreases in rates. In the present work, we thus adapt the statistical model developed and employed by \cite{eisenstein_diffusion_2014}. This model mediates the effect of these biases by controlling for spatial and temporal patterns in the rate at which data is obtained and by smoothing rate estimates via an autoregressive model. While a plethora of issues still must be considered when analyzing social media \citep{tufekci_big_2014} and news media \citep{} data, \citeapos{eisenstein_diffusion_2014} model allows us to move beyond these statistical irregularities in the data and thus gives us more freedom to draw inferences about the relationship between these adjusted rates of keyword usage and actual events occurring during the Arab Spring.

After explaining in more detail the methods utilized by \cite{eisenstein_diffusion_2014} and how they were adapted for the present work, we consider three analyses that address each of our three research questions.  With respect to RQ1, we find that \todoKenny{??}. With respect to RQ2, we find that \todoKenny{??}. With respect to RQ3, we find that \todoMatt{??}.

%From Ghita: if you talk about arab you don’t mean iran

%These concerns include the fact that ``data-driven'' analysis often suffers from post-hoc conclusions which substantially increase the number of ``researcher degrees of freedom'', even if unintentionally, in the analysis \cite{}. 

	%\item {\bf H3}: (term network) if media representation shows protestors as representative of the whole society, rather than as one particular group seeking partisan advantages for itself, revolution will be more successful
 	%\item Attempt to quantify how Mohammad Bouazizi’s setting himself on fire spreads across normally disconnected factions within the Twitter data in Tunisia from Dec 2010 – Feb 2011.
 	%\item I think we could use our religiosity terms to measure factionalism/sectarianism at the onset of Egypt’s protests.  Both counts and term networks could be useful.

%
%\subsection{My list of covariates}
%
%\begin{enumerate}
%    \item {\bf Regime-based}
%        \begin{enumerate}
%            \item religious freedom indices
%            \item personalist/not \cite{goldstone_bringing_2013}
%            \item wealth (particularly oil) \cite{goldstone_bringing_2013}
%            \item Twitter/News mentions of regime (of some kind) - sentiment?
%            \item Twitter/News revolution/insurgent/violence or adaptation terms
%        \end{enumerate}
%    \item {\bf International relations} 
%        \begin{enumerate}
%            \item international network position
%            \item international Twitter/News network position
%        \end{enumerate}
%    \item {\bf Population indicators } 
%        \begin{enumerate}
%            \item Twitter/news geospatial spread
%            \item Religious indices
%            \item Ethnic diversity / group indicators
%            \item Twitter/news mentions of ethnic group
%            \item Twitter/news attention / network position
%            \item Twitter/news ``class'' cohesiveness \cite{goldstone_cross-class_2011}
%            \item Twitter/news ``class'' spread \cite{goldstone_cross-class_2011}
%        \end{enumerate}
%\end{enumerate}
%
%

\section{Introduction}

Much has been made of the use of social media during the Arab Spring. At this point, it is almost common knowledge that the emphasis on social media as a \emph{cause} of the revolutions is overblown \citep{bruns_arab_2013}. However, many scholars have pointed to the fact that the use of social media may have aided certain aspects of the revolutions in important ways for different people  \citep{galle_who_2013,starbird_how_2012,,papacharissi_affective_2012} and that data from social media may help to understand how the revolutions grew and spread \citep{lotan_revolutions_2011,bruns_arab_2013}. By the same token, recent work has suggested that data from news media during the time of the Arab Spring may have itself been indicative of important developments during that time period \cite{joseph_arab_2014,pfeffer_rapid_2012}.

In the present work we investigate the extent to which news media and social media data, specifically from Twitter, correlate with publically avaialable indicators of political stability, religiosity and social change during the Arab Spring. Our work provides a significantly more detailed view of the conclusions that can be reached via the analysis of both social media and newspaper data, and the relation of these indicators to what actually occurred on the ground during the Arab Spring and the underlying social and political climates of the region. 

We consider the relationship between social media indicators, news media indicators and publically available, expert-driven indicators at three different ``levels'' of analysis. These three levels are increasingly detailed in their consideration of the variety of geotemporal, social and political covariates across which one can consider the data.  As we will show, we find that only particularly nuanced considerations of these covariates leave an informative picture of the underlying variables we treat as ground truth.  This finding suggests that high-level analyses of things like the number of tweets from a given country and tweets relevant to a particular word are only useful after careful consideration of the variety of biases inherent in this sort of data and of other important covariates.



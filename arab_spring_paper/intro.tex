\section{Introduction}
\IEEEPARstart{O}{n} December 17th, 2010, Mohamed Bouazizi immolated himself in Sidi Bou Said, Tunisia in response to harassment from a local policewoman and local municipality officers. Though Bouazizi was not the first to engage in this form of protest, for some reason his case resonated with other Tunisians, who took to the streets in protest of constant persecution and victimization by a corrupt government. Although early protests were relatively small and were met with violence by government forces, social media sites like Twitter, Facebook and YouTube recorded the events and displayed them to the broader public. These events are widely considered to be the beginning of what has come to be known, for better or worse \citep{gelvin_arab_2015}, as the Arab Spring.

There is little doubt that social media, and new media \cite{baym_personal_2010} more generally, played an important role in Arab Spring. However, conventional wisdom that emphasized social media as the cause of \emph{the} revolutions has been proven to be overblown \citep{bruns_arab_2013,goldstone_bringing_2013,comunello_will_2012}. Recent research has focused instead on how social media may have aided certain aspects of the revolutions in important ways for different people  \citep{galle_who_2013,starbird_how_2012,lotan_revolutions_2011,tufekci_social_2012}, and whether social processes that were carried out via new media are reflective of those that occurred ``offline'' \citep{comunello_will_2012}. Additionally, data from newspaper articles written during the time of the Arab Spring also may be of use in better understanding these processes better (REMOVED FOR BLIND REVIEW)%\citep{joseph_arab_2014,pfeffer_rapid_2012}
. Thus, social media and news media coverage should be appreciated as both pieces and reflections of a complex system of causal structures at play. 

Prior work on this subject has largely considered how the news media or Twitter are useful in understanding the social processes at play during particular events across a small set of countries \citep[e.g.][]{lotan_revolutions_2011,borge-holthoefer_content_2014} or a series of events in a particular country, most often Egypt \citep[e.g.][]{tufekci_social_2012}.  Further, few studies have considered traditional news sources and Twitter data side-by-side, leaving questions as to the similarities and differences in how media responded to, or possibly influenced, long and short term social processes during the Arab Spring. 

In the present work, we use a corpora of approximately 70M tweets and around 700K newspaper articles to provide an initial overview of the change in topical focus over time in sixteen countries relevant to the Arab Spring in both news and Twitter.  We take a breadth-over-depth approach, attempting to reconcile patterns in Twitter usage and news media coverage over a wide range of countries and multiple time periods. Additionally, we employ the same methodology for both datasets, allowing us to compare results across social and print media.  Specifically, we have focused on the following two research questions:
\begin{itemize}
	\item {\bf RQ1:} How did the topical focii of our news and Twitter data differ over time and across different countries and categories?
	\item {\bf RQ2:} Did significant changes in focus on different topics reflect on-the-ground processes?
\end{itemize}
In addition to these research questions, we provide a case study that explores how changes in the topical foci of news and Twitter users may be useful in understanding how topical discussions clustered around particular countries in interesting ways, especially with respect to discussions of protests.

In order to perform our analysis, we begin by developing a set of human-curated topical themes of interest based on a review of the literature. For each theme, we determined a set of terms that, when mentioned, were relevant to these themes. We searched for these terms across all of our Twitter and newspaper data. Where we found a term used in a particular tweet or news article, we determined the time at which the content was produced and the particular country(ies) in the Arab World to which the content was relevant. We were left with a set of counts, over time, of the discussion of our different themes in different countries in the Arab world.

In theory, this count data, or rates directly calculated from it, could be used to address our research questions.   As detailed in recent work by \cite{eisenstein_diffusion_2014}, however, the direct utilization of count data, or rate data based on these counts, is a methodologically unsound decision. As \cite{eisenstein_diffusion_2014} discuss with regards to Twitter, term counts may be biased by unknown irregularities in the way Twitter provides tweets through its API \citep{morstatter_is_2013} or via unique properties of the keyword or spatial queries researchers construct to obtain data from the API (REMOVED FOR BLIND REVIEW)%\citep{joseph_approach_2014
. Similarly, superfluous coverage by news media on specific countries may lead to artificial increases in counts or, if focused on themes not of interest for a study lead to superfluous decreases in rates.  In the present work, we adapt a slightly modified version of the statistical model developed and employed by \cite{eisenstein_diffusion_2014}. This model controls for spatial and temporal patterns in the rate at which data is obtained, thereby removing many of the important biases associated with term count data. While a plethora of issues must be considered when analyzing, in particular, social media data \citep{tufekci_big_2014}, \citeapos{eisenstein_diffusion_2014} model allows us to move beyond the statistical irregularities in the data and gives us more freedom to draw inferences about the relationship between rates of change in thematic content and actual events occurring during the Arab Spring.

After explaining this model in greater detail and how it was adapted for the present work, we consider analyses that address our two primary research questions, and then discuss results from the case study.  With respect to RQ1, we find that Twitter and traditional news media were more highly correlated on certain topics and in certain countries than others. More precisely, topics relevant to social change and in countries in which massive social change occurred showed high levels of correlation between the social and print media, while others showed significantly less cohesion in topical focus across the two media. With respect to RQ2, we find that outlier data, as determined by our model, matched quite well with important time periods during the Arab Spring, providing further evidence that news media and Twitter data are important tools for the study of social change. Finally, with respect to our case study, we find evidence that countries clustered in thematic discussions along dimensions of social change, i.e. that countries with more similar levels of social change appear to have more similar levels of discussion across the themes we studied. We also see evidence that temporal patterns in discussions of protest provide initial support for qualitative claims made in the political science and communications literature on news media, Twitter and protests during the Arab Spring.

%From Ghita: if you talk about arab you don’t mean iran

%These concerns include the fact that ``data-driven'' analysis often suffers from post-hoc conclusions which substantially increase the number of ``researcher degrees of freedom'', even if unintentionally, in the analysis \cite{}. 

	%\item {\bf H3}: (term network) if media representation shows protestors as representative of the whole society, rather than as one particular group seeking partisan advantages for itself, revolution will be more successful
 	%\item Attempt to quantify how Mohammad Bouazizi’s setting himself on fire spreads across normally disconnected factions within the Twitter data in Tunisia from Dec 2010 – Feb 2011.
 	%\item I think we could use our religiosity terms to measure factionalism/sectarianism at the onset of Egypt’s protests.  Both counts and term networks could be useful.

%
%\subsection{My list of covariates}
%
%\begin{enumerate}
%    \item {\bf Regime-based}
%        \begin{enumerate}
%            \item religious freedom indices
%            \item personalist/not \cite{goldstone_bringing_2013}
%            \item wealth (particularly oil) \cite{goldstone_bringing_2013}
%            \item Twitter/News mentions of regime (of some kind) - sentiment?
%            \item Twitter/News revolution/insurgent/violence or adaptation terms
%        \end{enumerate}
%    \item {\bf International relations} 
%        \begin{enumerate}
%            \item international network position
%            \item international Twitter/News network position
%        \end{enumerate}
%    \item {\bf Population indicators } 
%        \begin{enumerate}
%            \item Twitter/news geospatial spread
%            \item Religious indices
%            \item Ethnic diversity / group indicators
%            \item Twitter/news mentions of ethnic group
%            \item Twitter/news attention / network position
%            \item Twitter/news ``class'' cohesiveness \cite{goldstone_cross-class_2011}
%            \item Twitter/news ``class'' spread \cite{goldstone_cross-class_2011}
%        \end{enumerate}
%\end{enumerate}
%
%

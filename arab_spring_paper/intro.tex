\section{Introduction}

Story concepts:

how do discussions change over time and space in the arab spring?

this allows us to compare public focus and news focus

which drives which and how much

in intro: problems with count data, e.g. by showing diffrences in A, other biases

Work to do:

1. Fake data
	create 1 that is positive, does it recover
	random walk data

2. Stop word data

3. Re-estimate model with this	

4. Is A term significant? (Raw MLEs) Hyp A term is diff from 1; compare with boring data. Expected: A is negative over time; generally decreasing interest

5. news is fairly stable, Twitter is not. Also some countries with more noise less stable (bigger sigma?)

6. where do news and twitter diverge?  lag plot and just straight time series correlations; maybe think about indices for freedom of press

7. Overall, some high level comparison - event data; corr matrix w/ static data; ``event/outlier  detection'' (or, better, figure out how to get away from this)

8. Case study, Matt

9. lit review

Ideas not for this paper:
	correlate with network changes over time, compare/correlate categories, 


\citeapos{goldstone_cross-class_2011} recent assessment of the drivers of instability and revolutionary success during what has come to be known, for better or worse \cite{gelvin_arab_2015}, as the Arab Spring, suggest that a recurrent causal story played a role in the early revolutionary successes in Egypt, Tunisia and Libya.  Goldstone states that ``virtually all successful revolutions were forged by cross-class coalitions...pitting society as whole against the regime''. He then provides qualitative evidence that the revolutions in three countries were no different.  In Egypt, Tunisia and Libya, previously disparate social groups combined under a united ``cross-class coalition'' of protesters and revolutionaries, which made the revolutions significantly more likely to succeed.

In later work, \citeauthor{goldstone_bringing_2013} \citeyearpar{goldstone_bringing_2013} suggests that several additional predictors were important in determining whether or not a particular government was ultimately overthrown.  Chief amongst these predictors, he argued, was the structure of the overarching regime had a strong impact on whether or not the government was ultimately overthrown.  Goldstone defines a \emph{personalist} regime as one in which a single individual – who may have begun as an elected leader, or head of a military or even party regime – takes total or nearly total control of the national government. He then provides qualitative evidence for his believe that ``the single best key to where regimes in MENA have been overturned or faced massive rebellions is where personalist regimes have arisen''.
 
Goldstone's critiques are two of many articles to consider what may have led to the revolutions that spread throughout the MENA region and whose effects still continue to reverberate globally today. One factor of primary interest was the use of social media, which was originally suggested by popular media to be the sole cause of the spread of the revolution. At this point, it is almost common knowledge that this emphasis on social media as a \emph{cause} of the revolutions is overblown \citep{bruns_arab_2013,goldstone_bringing_2013,comunello_will_2012}. However, many scholars have pointed to the fact that the use of social media may have aided certain aspects of the revolutions in important ways for different people  \citep{galle_who_2013,starbird_how_2012,lotan_revolutions_2011,tufekci_social_2012}.  This fits with the current understanding of the causal structures involved,  as scholars now suggest that a breadth of heavily intertwined issues came together in a unique and unpredictable fashion to provide the conditions necessary for the spread of both successful and unsuccessful revolt and protest throughout the region. 

In contrast to work considering the causal role that social media played in the revolution, recent work has suggested that data from social media data may help to inform this broader understanding of how the revolutions grew and spread \citep{}.  This use of social media data takes as a given that these tools were used in some way during the revolution and goes beyond this in assuming, via the well-worn notion of media multiplexity \citep{}, that causal processes of the revolution which existed ``offline'' are likely to have played out ``online'' as well.  Such work has shown that, for example, X, Y, Z. Similarly, recent work has suggested that data from news media during the time of the Arab Spring also may be of use in better understanding these processes \citep{joseph_arab_2014,pfeffer_rapid_2012}. Combined, efforts with social and news media data from the Arab Spring have shown the value in using this data to a) provide evidence for or against prior hypotheses presented in the literature using quantitative metrics on large sets of data, and b) use the available data to generate new or more nuanced understandings of the incredibly complex web of factors that have been implicated in a causal role.

The present work provides an example of how social and news media data can be used in each of these two ways. First, we extract quantitative evidence for the assertions of \citeapos{goldstone_cross-class_2011}. More specifically, we consider the following two claims made by Goldstone and consider the extent to which they played out in social and news media across Egypt, Libya and Tunisia as well as fourteen other nations in the MENA region:
\begin{itemize}
	\item {\bf H1}: \emph{If a country has more disparate groups, it is more unstable}
	\item {\bf H2}: \emph{If these groups align under a revolutionary cause, the revolution is more likely to succeed}
\end{itemize}

These two hypotheses are directly quantifiable via a combination of social media and news data and external indicators based on expert opinion.  \todoKenny{In order to evaluate {\bf H1}, we consider ... In order to evaluate {\bf H2}, we ...}

In the second part of the paper, we provide a characterization of how social media use may have differed in nations with personalist regimes as opposed to nations with other types of regimes discussed by Goldstone. We combine a host of metrics measured on both news and social media and use a variety of statistical techniques to extract differences.  \todoKenny{We find XYZ. This leads to an interesting extension/correlary/etc of Goldstone in that XYZ...}

%From Ghita: if you talk about arab you don’t mean iran

%These concerns include the fact that ``data-driven'' analysis often suffers from post-hoc conclusions which substantially increase the number of ``researcher degrees of freedom'', even if unintentionally, in the analysis \cite{}. 

	%\item {\bf H3}: (term network) if media representation shows protestors as representative of the whole society, rather than as one particular group seeking partisan advantages for itself, revolution will be more successful
 	%\item Attempt to quantify how Mohammad Bouazizi’s setting himself on fire spreads across normally disconnected factions within the Twitter data in Tunisia from Dec 2010 – Feb 2011.
 	%\item I think we could use our religiosity terms to measure factionalism/sectarianism at the onset of Egypt’s protests.  Both counts and term networks could be useful.

%
%\subsection{My list of covariates}
%
%\begin{enumerate}
%    \item {\bf Regime-based}
%        \begin{enumerate}
%            \item religious freedom indices
%            \item personalist/not \cite{goldstone_bringing_2013}
%            \item wealth (particularly oil) \cite{goldstone_bringing_2013}
%            \item Twitter/News mentions of regime (of some kind) - sentiment?
%            \item Twitter/News revolution/insurgent/violence or adaptation terms
%        \end{enumerate}
%    \item {\bf International relations} 
%        \begin{enumerate}
%            \item international network position
%            \item international Twitter/News network position
%        \end{enumerate}
%    \item {\bf Population indicators } 
%        \begin{enumerate}
%            \item Twitter/news geospatial spread
%            \item Religious indices
%            \item Ethnic diversity / group indicators
%            \item Twitter/news mentions of ethnic group
%            \item Twitter/news attention / network position
%            \item Twitter/news ``class'' cohesiveness \cite{goldstone_cross-class_2011}
%            \item Twitter/news ``class'' spread \cite{goldstone_cross-class_2011}
%        \end{enumerate}
%\end{enumerate}
%
%
